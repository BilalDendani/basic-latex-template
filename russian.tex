\documentclass[12pt,a4paper]{article}
\usepackage[margin=1.25in]{geometry}
\usepackage[utf8]{inputenc}
\usepackage{fancyhdr} % fancy header
\pagestyle{fancy} % so fancy
\usepackage[russian]{babel} % for russian letters

\lhead{Joshua MEYER}
\rhead{Сочинение о Переходном Возросте}
\cfoot{} %% make empty to get rid of the page number %% \cfoot{Page \thepage}
\renewcommand{\footrulewidth}{0.4pt} %% this puts a fancy line at the footer


\begin{document}

Основываясь на данных Википедии, переходный возрост человека начинается с 14и лет.
Этот период жизни ассосируется с разностороннами переменами оргамизма, в томе числе
и физические и психологические изменения. Вследствие этих переменов, средний подросток
столкивается с проблемами в школе, дома, и в обществе. В школе у преподавателей есть свои правила,
дома у родители есть свои, и в обшестве правила сохраняются в форме законов. Пытаясь понять себя
лучше, подростки переходного возроста нарушают эти правила даны им, чтобы расширить границы возможного.\\

В каждом обшестве, ход развития переходного возроста образуется по разному. Лично, у меня было
опыт этого возроста в средном западе США, точнее в городе Омаха, в штате Небраска. В этих годах,
я учился в католической школе, и обстоятельства в этой школе знаменательно влияли на мой опыт. \\

Когда мне было 14 лет, я посутпил в старшие классы школы, в католическую школу где учится только мальчики.
Не было у нас обшежития, но мы провели каждый день вместе с 7.30 до 3.30. Днем у нас были занятия,
а по окончании уроков кому-то тренеровки, кому-то работа, и кому-то просто домой.
Мы изучали разные дисциплины, как и обшеобразовательные, так и католические. В нашем городе такое образование
считается нормальным явлением. В городе у нас и религиозные школы, и светские, и мы как студенты часто проводили вместе время.
Чаше всего, мы сидели в парках. Средный запад америки малонаселен. Поэтому, у подростков есть много мест
где можно спокойно погулять. В Омахе у нас много большых, зеленых, цветуюших парков. Именно там обсуждали
и наши мечты о будущем и наши ностоящие проблемы, и пытались понять самих себя.\\

Тогда, много непонятных нам обстоятельств окружили нас, но, спустя 10 лет, получили какое-то освещение, какое-то
озарение, и какое-то принятие.\\

Среди этих озарении, один выделается - то что наши друзья из светских школ столкнулись с совсем другими проблемами. Да,
между нами было много общего, но все равно сушествовались большие разницы. Мы все хотели наидти наше место в
обшестве, как и все подростки в мире, но в религиозных заведениях обучения, ограничения имеют другой вкус.\\

Наша школа славится успехом студентов - и действительно наши выпустники встретят успехи в разных сферах рабочей жизни.
Тем не менее, для учающего подростка этой школы, наидутся проблемы с чем придется бороться.\\

К сожалению, такая школа питает элитизм в форме сексизма, расизма, и гомофобии. Эти проблемы не только связаны с тем,
что школа является религиознимой, но и с тем, что в таких частних школах учятся дети боготых семей. Когда дети самых боготых
семей только между собой общаются, они не понимают проблемы и мнение других. И более того, часто люди сразу
воспринимают то что не понимают как отклоняющеемся, и следовательно, плохим.\\

В течение переходный период, студенты принадлежащие к меньшинственной группой испытывают больше расстройства, беспокойство,
страха, и депрессии чем среднего студента. Бороться с мнением родителей, преподавателей, и власти - это одно,
но бороться с мнением ровесников (и в том числе друзьей) - это другое. Менталитет коллектива сложно распознать таковым
когда находишся среды коллектива. Когда у большинства твоих знакомых есть похожие взглядыи, взгляды становятся обыкновеннами.
И в основном, мы все хотим быть одним из группы, потому что страшно быть один. Таким путем, студенты которые не согласны
с этими взглядами либо их принимают, либо становиться одиноками.Чернокожые студенты всегда вместе сидели на обед, гей студенты
боялись совершать каминг-аут, и бедние студенты на стипенды скривались их экономическое положение. Такие студенты редко
принадлежали к кругам боготых студентов. \\

Что касается религии, наше образование, и консерватизм нашего региона, это еще одно слоя на проблеме. Подростки переходного возроста
нарушят правила обшества, и когда есть много правил, тогда будут много нарушения. Чем больше взрослые
пытаются присматривать и контролировать детей, тем больше дети бутуют. Такая было ситуация у нас в школе. Много студентов
начали пить когда еще малолетними, и потом уже употреблять наркотики. Более того, из за того что имеем права водить
машину когда нам 16 лет, и хотим скрывать от родителей чем занимаемся незаконого, много студентов пьют и ездят. В нашем городе,
из за того что обшество так жестко реагирует на употребление акоголь и наркотики, когда студенты употребляют, они обично
перебарщивают, потому что знают что количество употребления не имеет значение для закона - если тебе 16 лет и пил одно пиво,
наказание будет так же, как если ты выпил бы бутилку водки. Поэтому, есть склонность к переборе. \\


У каждого подростка есть период переходного возроста, и всегда будут ему трудности. Родители хотят самое лучшее для их детей - не хотят что бы дети совершили те же ошибки как они. Наидут лучшее, самые престижные школы для их сыновей, для их дочерей. Пытаются показать правильный путь. Но, тем не менее, проблемы возникают из за того что подросткам не важно какие ограничение есть, все равно нарушят правила. И скорее всего, чем строжее правила, тем больше рождается бунтарь в подростке. 

\end{document}



 
