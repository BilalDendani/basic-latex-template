\documentclass[12pt,a4paper]{article}
\usepackage[margin=1.25in]{geometry}
\usepackage{fancyhdr} % fancy header
\pagestyle{fancy} % so fancy
\usepackage[russian,english]{babel} % for russian letters
\usepackage{tipa} % for IPA symbols
\usepackage[round]{natbib} % bibliography
\usepackage{graphicx} % for importing graphics / figures
\usepackage{booktabs} % publication-worthy tables
\usepackage{adjustbox} % makes tables fit nicely on the page
\usepackage{tikz}

\def\checkmark{\tikz\fill[scale=0.5](0,.35)--(.25,0)--(1,.7)--(.25,.15)--cycle;}

\lhead{Joshua MEYER}
\rhead{Survey of Language Technologies in Central Asia}
\cfoot{} %% make empty to get rid of the page number %% \cfoot{Page \thepage}
\renewcommand{\footrulewidth}{0.4pt} %% this puts a fancy line at the footer


\begin{document}


\section{Introduction}

There exist two main kinds of language technology: tools for (1) text processing and (2) audio processing.

There are many applications of either of these, and most audio processing requires text processing as well.

\begin{enumerate}
\item Text Processing
  \begin{enumerate}
  \item Spell-check (Microsoft Word)
  \item Word prediction (Smartphones)
  \item Part of Speech tagging
  \end{enumerate}

\item Audio Processing
  \begin{enumerate}
  \item Speech Recognition
    \begin{enumerate}
    \item Automatic subtitle generation (Youtube)
    \item Telephone call transcription
    \item Personal Assistant interaction (Siri, Ok Google)
    \end{enumerate}

  \item Speech Synthesis
    \begin{enumerate}
    \item Screen Readers (NVDA)
    \item Automated phonecalls
    \item Personal Assistant interaction (Siri, Ok Google)
    \end{enumerate}
    
  \end{enumerate}

\end{enumerate}


\section{Existing Technologies}

Given the millions of speakers of Central Asian languages, the current lack of language technologies is noticable.

\begin{table}[htbp]
    \caption{Current open-source language technologies}
  \centering
  \begin{adjustbox}{width=.95\textwidth}
    \begin{tabular}{lccccc}
      \toprule
      \textbf & \multicolumn{2}{c}{\textbf{Speech Synthesis}} & \multicolumn{2}{c}{\textbf{Speech Recognition}} \\
      \toprule
      \textbf{Language} &  \textsc{eSpeak} &  \textsc{Merlin} & \textsc{CMU Sphinx} & \textsc{Kaldi}\\
      \midrule
      \textsc{Kyrgyz} & \checkmark & \checkmark & & \checkmark   \\
      \textsc{Kazakh} & \checkmark & & \checkmark  \\
      \textsc{Tajik} &  &  &  \\
      \textsc{Tatar} & \checkmark &  & \\
      \textsc{Turkmen} &  &  &  \\
      \textsc{Uzbek} &  &  &  \\
      \bottomrule
    \end{tabular}
    \label{table:cool_table}
  \end{adjustbox}
\end{table}



\section{Missing Data}

One of the main reasons for the lack of these speech technologies is that they require large collections of speech to be created, and those databases currently don't exist.

Namely, there are three main kinds of speech which need to be collected and transcribed:

\begin{enumerate}
\item Television / Film data
\item Telephone conversations
\item Audiobooks
\end{enumerate}

For speech synthesis, the audiobooks are usually more easy to find (example for Kyrgyz: bizdin.kg). Here, even with 2 hours of speech we get decent results.

However, the other two sources for speech recognition are more difficult to find and require more hours of audio. The standard models used in production for English use $2,000$ hours of audio.

\section{Future Directions}

To create any kind of decent speech technology, we need to collect those resources.

Governments and universities have traditionally been helpful in this, but businesses have much to gain and can invest.




\end{document}



 
